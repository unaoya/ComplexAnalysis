\documentclass{ltjsarticle}
\title{2024年度複素解析I(金曜3限)ガイダンス}
\author{梅崎直也}

\begin{document}
\maketitle

\section*{授業について}

この授業では、複素数を変数とし複素数に値を持つような複素関数の微分積分について解説します。
詳しい内容については授業のシラバスを見てください。

今回(第1回)はこの資料を読んでいただければ十分です。
必要に応じて微分積分、例えば、指数関数や対数関数、テイラー展開、べき級数、偏微分、全微分、広義積分、線積分などを復習しておいてください。
次回以降は教室で対面授業を行います。
みなさまにお会いできるのを楽しみにしています。

授業では数学的な内容の解説が中心になりますが、数学を理解するためには問題演習が欠かせません。
この授業の内容に対応する問題演習の授業が「複素解析I演習」という科目で開講されます。
こちらの演習の授業も併せて履修することを強くお勧めします。

\section*{成績について}

基本的には期末試験の成績で評価します。全員忘れずに受験してください。
病欠などのやむを得ない理由があり、それが認められた場合には、追試験を受けることができます。
これについては期末試験が近づいてから説明します。

期末試験後のレポートなどによる救済は一切しません。
試験問題はこの授業で学んだことと、その応用から出題します。
「複素解析I演習」の問題が自分の力で解けるようになっていれば単位は取得できるような試験にします。

参考に昨年度の試験問題を公開します。
また授業の最終回に試験に向けた演習問題を配布します。
昨年度のものを参考にしてください。

\section*{質問について}

この授業の内容に関する質問は、この授業中と授業前後に受け付けます。
特に、授業中は他の学生の方のためにもなるので、ぜひ積極的に質問してください。

私は非常勤講師で、キャンパスにはこの授業の時間帯しか来ません。
通常は授業前後に質問対応可能ですが、どうしても質問する必要があるときは、事前にメールでアポイントを取ってください。
メールアドレスはumezakinaoya@gmail.comです。

できるだけ見落としを防ぐために、メールを送る際は件名を「青山学院複素解析」としてください。
これは質問以外で連絡をとる場合も同様です。

\section*{教科書や参考書について}

教科書の指定はありません。
授業中の板書をノートに取っていただいたものが教科書となります。
きちんと出席し、適宜復習をしていただければ十分演習や試験に対応できるようにします。

参考書はシラバスにある通りですが、特に参照せずとも理解できるように授業をします。

こちらのページも参考にしてください。
昨年の同じ授業を行なったものをもとにしたノートです。
随時更新しますので、印刷などする際はご注意ください。

YouTubeのこちらの動画は、講義で扱う内容そのものではないのですが、複素解析のイメージを掴むことができると思うので、
ご興味あればご覧ください。
ゼータ関数の見た目【解析接続】
%https://www.youtube.com/watch?v=Xjja6Cc7lio&t=2s



\end{document}