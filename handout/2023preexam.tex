\documentclass{ltjsarticle}
\RequirePackage{amsmath,amssymb,amsthm, amscd, comment, multicol, ascmac}
\usepackage{tikz}
\usepackage{graphicx}
\usepackage{multicol}
\usepackage{bookmark}
\usepackage{xurl}
\hypersetup{unicode,bookmarksnumbered=true,hidelinks,final}
\usepackage[top=10truemm,bottom=20truemm,left=25truemm,right=25truemm]{geometry}
\newtheorem{prob}{問題}
\newtheorem{ans}{解答}

\title{期末試験向け練習問題}
\date{}

\begin{document}
\maketitle
\vspace{-13mm}

\begin{prob}
    複素数$z$を実数$x, y$を用いて$z=x+yi$と表し、
    複素数値関数$f(z)$の実部を$u$、虚部を$v$とし$f(z)=u(x,y)+iv(x,y)$と表す。
    \begin{enumerate}
        \item この$f$についてコーシーリーマンの方程式を書け。
        \item 以下の関数$f$が正則であるかどうかを答えよ。
        \begin{multicols}{2}
            \begin{enumerate}
                \item $f(z)=x^2-y^2+2xyi$
                \item $f(z)=x^2-y^2-2xyi$
            \end{enumerate}            
        \end{multicols}
    \end{enumerate}
\end{prob}

\begin{prob}
    次の線積分の値を求めよ。
    ただし$C(z,r)$で反時計回りの中心$z$半径$r$の円周を表す。
    \begin{multicols}{3}
        \begin{enumerate}
            \item $$\int_{C(0,1)}\frac{z}{z+2}dz$$
            \item $$\int_{C(1,1)}\frac{1}{z^2-1}dz$$
            \item $$\int_{C(0,1)}\frac{\sin z}{z^2}dz$$
        \end{enumerate}
    \end{multicols}
\end{prob}

\begin{prob}
    次の$z=0$を中心とする冪級数の収束半径を求めよ。
    \begin{multicols}{3}
        \begin{enumerate}
            \item $$\sum_{n=1}^\infty\frac{1}{n}z^n$$
            \item $$\sum_{n=0}^\infty\frac{1}{n!}z^n$$
            \item $$\sum_{n=1}^\infty\frac{1}{2^n}z^{2n}$$
        \end{enumerate}
    \end{multicols}
\end{prob}

\begin{prob}
    次の関数$f$の$z=0$を中心とするローラン展開を$3$次以下まで求めよ。
    \begin{multicols}{2}
        \begin{enumerate}
            \item $f(z)=\dfrac{1}{z-2}$
            \item $f(z)=\dfrac{e^z}{z^2}$
        \end{enumerate}
    \end{multicols}
\end{prob}

\begin{prob}
    次の関数$f$は$\mathbb{C}$の有理型関数である。
    この関数の極とその位数および留数を全て求めよ。
    \begin{multicols}{2}
        \begin{enumerate}
            \item $f(z)=\dfrac{1}{z(z+1)}$
            \item $f(z)=\dfrac{e^z-1}{z^2(z-1)^2}$
        \end{enumerate}
    \end{multicols}
\end{prob}

\begin{prob}
    以下の積分値を留数定理を用いて求めよ。
    ただし、積分経路の取り方、留数定理をどう使うか、収束についての議論などを必要に応じて説明せよ。
    \begin{multicols}{2}
        \begin{enumerate}
            \item $\displaystyle\int^{2\pi}_0\frac{1}{5+4\cos\theta}d\theta$
            \item $\displaystyle\int^\infty_{-\infty}\frac{1}{x^4+1}dx$
        \end{enumerate}
    \end{multicols}
\end{prob}

\newpage
\begin{ans}
    \begin{enumerate}
        \item 
        \begin{align*}
            \begin{cases}
                \dfrac{\partial u}{\partial x}=\dfrac{\partial v}{\partial y}\\
                \dfrac{\partial u}{\partial y}=-\dfrac{\partial v}{\partial x}
            \end{cases}
        \end{align*}
        \item 
        \begin{enumerate}
            \item 実部虚部ともに$x, y$の多項式なので全微分可能である。
            また、$\dfrac{\partial u}{\partial x}=2x=\dfrac{\partial v}{\partial y},
            \dfrac{\partial u}{\partial y}=-2y=-\dfrac{\partial v}{\partial y}$なので、コーシーリーマン方程式を満たす。
            よって$f$は正則関数である。
            \item $\dfrac{\partial u}{\partial x}=2x\neq\dfrac{\partial v}{\partial y},
            \dfrac{\partial u}{\partial y}=-2y\neq-\dfrac{\partial v}{\partial y}$なので、コーシーリーマン方程式を満たさない。
            よって$f$は正則関数でない。
        \end{enumerate}
    \end{enumerate}
\end{ans}

\begin{ans}
    \begin{enumerate}
        \item $\dfrac{z}{z+2}$は$C(0,1)$上および内部で正則である。よって、コーシーの積分定理により、積分値は$0$である。
        \item $f(z)=\dfrac{1}{z+1}$とすると、これは$C(1,1)$上および内部で正則なので、コーシーの積分公式により、
            \begin{align*}
                \int_{C(1,1)}\frac{1}{z^2-1}dz&=\int_{C(1,1)}\frac{1}{z-1}f(z)dz\\
                &=(2\pi i)f(1)=\pi i
            \end{align*}
            である。
        \item $f(z)=\sin z$とすると、これは$C(0,1)$上および内部で正則なので、導関数を積分で表す公式により
            \begin{align*}
                \int_{C(0,1)}\frac{\sin z}{z^2}dz&=\int_{C(0,1)}\frac{f(z)}{z^2}dz\\
                &=2\pi if'(0)=2\pi i
            \end{align*}
            である。
    \end{enumerate}
\end{ans}

\begin{ans}
    \begin{enumerate}
        \item \begin{align*}\lim_{n\to\infty}\frac{1/(n+1)}{1/n}=1\end{align*}であるので、収束半径は$1$である。
        \item \begin{align*}\lim_{n\to\infty}\frac{1/(n+1)!}{1/n!}=0\end{align*}であるので、収束半径は無限大である。
        \item \begin{align*}a_n=\begin{cases}\dfrac{1}{\sqrt{2}^n}&n\mbox{が偶数}\\0&n\mbox{が奇数}\end{cases}\end{align*}とおくと、
            $\displaystyle\varlimsup_{n\to\infty}\sqrt[n]{\lvert a_n\rvert}=\dfrac{1}{\sqrt{2}}$なので、収束半径は$\sqrt{2}$である。
    \end{enumerate}
\end{ans}

\begin{ans}
    \begin{enumerate}
        \item 
            \begin{align*}
                \frac{1}{z-2}&=-\frac{1}{2}\frac{1}{1-(z/2)}\\
                &=-\frac{1}{2}(1+\frac{z}{2}+\frac{z^2}{4}+\frac{z^3}{8}+\cdots)\\
                &=-\frac{1}{2}-\frac{1}{4}z-\frac{1}{8}z^2-\frac{1}{16}z^3+\cdots
            \end{align*}
        \item
            \begin{align*}
                \frac{e^z}{z^2}&=\frac{1}{z^2}(1+z+\frac{1}{2}z^2+\frac{1}{6}z^3+\frac{1}{24}z^4+\frac{1}{120}z^5+\cdots)\\
                &=z^{-2}+z^{-1}+\frac{1}{2}+\frac{1}{6}z+\frac{1}{24}z^2+\frac{1}{120}z^3
            \end{align*}
    \end{enumerate}
\end{ans}

\begin{ans}
    いずれも正則関数の比なので、分母が$0$となる点が極の候補。
    ただし、分子が$0$になるかどうかに注意する。
    \begin{enumerate}
        \item $z=0$で$1$位の極をもち、留数は$\displaystyle\lim_{z\to0}zf(z)=1$である。
        また、$z=-1$で$1$位の極をもち、留数は$\displaystyle\lim_{z\to-1}(z+1)f(z)=-1$である。
        \item 分子は$z=0$での$0$となり、$z=0$の周りでテイラー展開すると$z+\dfrac{1}{2}z^2+\cdots$となるので、
        $f(z)$は$z=0$で$1$位の極をもち、留数は$\displaystyle\lim_{z\to0}zf(z)=1$である。
        また、$z=1$で$2$位の極をもち、$\displaystyle\lim_{z\to1}((z-1)^2f(z))'=2-e$である。
    \end{enumerate}
\end{ans}

\begin{ans}
    \begin{enumerate}
        \item $z=e^{i\theta}$とすると、$\cos\theta=\dfrac{z+z^{-1}}{2}, d\theta=\dfrac{dz}{iz}$であり、
        $\theta$についての積分区間が単位円$C(0,1)$での線積分になるため、
        \begin{align*}
            \displaystyle\int^{2\pi}_0\frac{1}{5+4\cos\theta}d\theta
            &=\int_{C(0,1)}\frac{1}{5+2(z+z^{-1})}\frac{dz}{iz}\\
            &=\frac{1}{i}\int_{C(0,1)}\frac{1}{2z^2+5z+2}dz\\
            &=\frac{1}{i}\int_{C(0,1)}\frac{1}{(2z+1)(z+2)}dz\\
            &=\frac{1}{i}(2\pi i)\mathrm{Res}_{z=-1/2}(\frac{1}{(2z+1)(z+2)})=\frac{4\pi}{3}
        \end{align*}
        \item 正の数$R>1$に対して、積分経路を原点中心で半径$R$の上半円を反時計回りに向きをつけたもの$C_R$と実軸の$-R$から$R$へ向かう線分$I_R$をつなげたものとする。
        これに対して、留数定理を用いると
        \begin{align*}
            \int_{C_R+I_R}\dfrac{1}{z^4+1}dz&=2\pi i\left(\mathrm{Res}_{z=e^{\pi i/4}}\dfrac{1}{z^4+1}+\mathrm{Res}_{z=e^{3\pi i/4}}\dfrac{1}{z^4+1}\right)\\
            &=2\pi i(\frac{1}{4e^{\pi i/4}}+\frac{1}{4e^{3\pi i/4}})=\frac{\pi}{\sqrt{2}}
        \end{align*}
        となる。
        一方で、$R\to\infty$での極限は、
        \begin{align*}
            \lim_{R\to\infty}\int_{I_R}\frac{1}{z^4+1}dz=\int_{-\infty}^{\infty}\frac{1}{x^4+1}dx
        \end{align*}
        であり、
        \begin{align*}
            \left\lvert\int_{C_R}\frac{1}{z^4+1}dz\right\rvert
            &\leq\int_0^\pi\left\lvert\frac{1}{R^4e^{4i\theta}+1}Rie^{i\theta}\right\rvert d\theta\\
            &\leq\frac{R}{R^4-1}\pi
        \end{align*}
        となるので、
        \begin{align*}
            \lim_{R\to\infty}\int_{C_R}\frac{1}{z^4+1}dz=0
        \end{align*}
        である。
        よって、
        \begin{align*}
            \int_{-\infty}^\infty\frac{1}{x^4+1}dx=\frac{\pi}{\sqrt{2}}
        \end{align*}
    \end{enumerate}
\end{ans}
\end{document}